%--------------------------------------------------------------------------------
%
%
%--------------------------------------------------------------------------------
\section*{Instruction Formats}
\addcontentsline{toc}{section}{Instruction Formats}

T64 uses few instruction formats. They are grouped by the number of available register slots and the length of the immediate value field.

\subsection*{Immediate S19}

The immediate S19 instruction format provides one register and a 19 bit signed immediate field. This format is typically used by branch instructions.

\begin{center}
    \begin{tikzpicture}[scale=0.9, transform shape]
        
        \draw[help lines, gray!70, dashed] (0,0) grid(16,1.5);
        
        \node[  tsRectangle, 
                minimum width=16cm,
                minimum height=1cm,
                text width=3cm,
                text centered,
                fill=white!50] (blocks) at (8,1) { };
                
     	\draw[-] (1,0.5) -- (1,1.5);
     	\draw[-] (3,0.5) -- (3,1.5);
     	\draw[-] (5,0.5) -- (5,1.5);
     	\draw[-] (6.5,0.5) -- (6.5,1.5);
     	
     	\node at (0.5,0.25) {2};
    	\node at (2,0.25) {4};
    	\node at (4,0.25) {4};
    	\node at (5.75,0.25) {3};
   	 	\node at (11,0.25) {19};
   	 	
   	 	\node at (0.5,1) {\small Grp};
		\node at (2,1) {\small OpCode};
		\node at (4,1) {\small Reg R};
		\node at (5.75,1) {\small Opt 1};
		\node at (11,1) {\small S-IMM-19};
   	 	
       \end{tikzpicture}
\end{center}

\subsection*{Immediate S15}

The immediate S15 instruction format provides two registers and a 15 bit signed immediate field. This format is used by branch instructions as well as instruction which operate on a register and an immediate value.

\begin{center}
    \begin{tikzpicture}[scale=0.9, transform shape]
        
        \draw[help lines, gray!70, dashed] (0,0) grid(16,1.5);
        
        \node[  tsRectangle, 
                minimum width=16cm,
                minimum height=1cm,
                text width=3cm,
                text centered,
                fill=white!50] (blocks) at (8,1) { };
                
     	\draw[-] (1,0.5) -- (1, 1.5);
     	\draw[-] (3,0.5) -- (3, 1.5);
     	\draw[-] (5,0.5) -- (5, 1.5);
     	\draw[-] (6.5,0.5) -- (6.5, 1.5);
     	\draw[-] (8.5,0.5) -- (8.5, 1.5);
          	
     	\node at (0.5,0.25) {2};
    	\node at (2,0.25) {4};
    	\node at (4,0.25) {4};
    	\node at (5.75,0.25) {3};
    	\node at (7.5,0.25) {4};
   	 	\node at (11.5,0.25) {15};
   	 	
   	 	\node at (0.5,1) {\small Grp};
		\node at (2,1) {\small OpCode};
		\node at (4,1) {\small Reg R};
		\node at (5.75,1) {\small Opt 1};
		\node at (7.5,1) {\small Reg B};
		\node at (11.5,1) {\small S-IMM-15};
   	 	
       \end{tikzpicture}
\end{center}

\subsection*{Immediate S13}

The immediate S13 instruction format provides two registers, an additional option field  and a 13 bit signed immediate field. Some instruction use the S-IMM-13 field for special encodings instead of a signed value. This format is used by memory reference instructions where the address is formed by a base register and a 13-bit signed offset.

\begin{center}
    \begin{tikzpicture}[scale=0.9, transform shape]
        
        \draw[help lines, gray!70, dashed] (0,0) grid(16,1.5);
        
        \node[  tsRectangle, 
                minimum width=16cm,
                minimum height=1cm,
                text width=3cm,
                text centered,
                fill=white!50] (blocks) at (8,1) { };
                
     	\draw[-] (1,0.5) -- (1, 1.5);
     	\draw[-] (3,0.5) -- (3, 1.5);
     	\draw[-] (5,0.5) -- (5, 1.5);
     	\draw[-] (6.5,0.5) -- (6.5, 1.5);
     	\draw[-] (8.5,0.5) -- (8.5, 1.5);
     	\draw[-] (9.5,0.5) -- (9.5, 1.5);
     	
     	\node at (0.5,0.25) {2};
    	\node at (2,0.25) {4};
    	\node at (4,0.25) {4};
    	\node at (5.75,0.25) {3};
    	\node at (7.5,0.25) {4};
    	\node at (9,0.25) {2};
     	\node at (12.25,0.25) {13};
   	 	
   	 	\node at (0.5,1) {\small Grp};
		\node at (2,1) {\small OpCode};
		\node at (4,1) {\small Reg R};
		\node at (5.75,1) {\small Opt 1};
		\node at (7.5,1) {\small Reg B};
		\node at (9,1) {\small Opt 2};
		\node at (12.25,1) {\small S-IMM-13 / special};
   	 	
       \end{tikzpicture}
\end{center}

\subsection*{Immediate S9}

The immediate S9 instruction format provides three registers, an additional option field  and a 9 bit signed immediate field. Some instruction use the S-IMM-9 field for special encodings instead of a signed value.instead of a signed value. This instruction format is used for all computational instructions where three registers are needed.

\begin{center}
    \begin{tikzpicture}[scale=0.9, transform shape]
        
        \draw[help lines, gray!70, dashed] (0,0) grid(16,1.5);
        
        \node[  tsRectangle, 
                minimum width=16cm,
                minimum height=1cm,
                text width=3cm,
                text centered,
                fill=white!50] (blocks) at (8,1) { };
                
     	\draw[-] (1,0.5) -- (1, 1.5);
     	\draw[-] (3,0.5) -- (3, 1.5);
     	\draw[-] (5,0.5) -- (5, 1.5);
     	\draw[-] (6.5,0.5) -- (6.5, 1.5);
     	\draw[-] (8.5,0.5) -- (8.5, 1.5);
     	\draw[-] (9.5,0.5) -- (9.5, 1.5);
     	\draw[-] (11.5,0.5) -- (11.5, 1.5);
     	
     	\node at (0.5,0.25) {2};
    	\node at (2,0.25) {4};
    	\node at (4,0.25) {4};
    	\node at (5.75,0.25) {3};
    	\node at (7.5,0.25) {4};
    	\node at (9,0.25) {2};
    	\node at (10.5,0.25) {4};
   	 	\node at (13.75,0.25) {9};
   	 	
   	 	\node at (0.5,1) {\small Grp};
		\node at (2,1) {\small OpCode};
		\node at (4,1) {\small Reg R};
		\node at (5.75,1) {\small Opt 1};
		\node at (7.5,1) {\small Reg B};
		\node at (9,1) {\small Opt 2};
		\node at (10.5,1) {\small Reg A};
		\node at (13.75,1) {\small S-IMM-9 / special };
   	 	
       \end{tikzpicture}
\end{center}


\subsection*{Immediate U20}

The immediate U20 format is used by the immediate instruction group to provide a 20bit unsigned value. This value is then placed at certain positions in the register target Reg R.

\begin{center}
    \begin{tikzpicture}[scale=0.9, transform shape]
        
        \draw[help lines, gray!70, dashed] (0,0) grid(16,1.5);
        
        \node[  tsRectangle, 
                minimum width=16cm,
                minimum height=1cm,
                text width=3cm,
                text centered,
                fill=white!50] (blocks) at (8,1) { };
                
     	\draw[-] (1,0.5) -- (1,1.5);
     	\draw[-] (3,0.5) -- (3,1.5);
     	\draw[-] (5,0.5) -- (5,1.5);
     	\draw[-] (6,0.5) -- (6,1.5);
     	     	
     	\node at (0.5,0.25) {2};
    	\node at (2,0.25) {4};
    	\node at (4,0.25) {4};
    	\node at (5.5,0.25) {2};
   	 	\node at (11,0.25) {20};
   	 	
   	 	\node at (0.5,1) {\small Grp};
		\node at (2,1) {\small OpCode};
		\node at (4,1) {\small Reg R};
		\node at (5.5,1) {\small 0};
		\node at (11,1) {\small U-IMM-20};
   	 	
       \end{tikzpicture}
\end{center}




